\documentclass{article}
\usepackage[margin=1in]{geometry}
\usepackage{amsmath}
\usepackage{hyperref}
\usepackage{graphicx}
\usepackage[export]{adjustbox}
\usepackage{caption}
\usepackage{listings}
\usepackage{xcolor}
\usepackage{color}
\usepackage{multirow}
\usepackage{lscape}
\usepackage{colortbl}
\usepackage{hanging}
\usepackage[most]{tcolorbox}
% \usepackage{titlesec}
% \usepackage{indentfirst}

\tcbset{
    frame code={}
    center title,
    left=0pt,
    right=0pt,
    top=0pt,
    bottom=0pt,
    colback=gray!30,
    colframe=white,
    width=\dimexpr\textwidth\relax,
    enlarge left by=0mm,
    boxsep=5pt,
    arc=0pt,outer arc=0pt,
    }

\parindent 0pt
\parskip 2ex % paragraph spacing

\hypersetup{
    colorlinks,
    citecolor=red,
    filecolor=green,
    linkcolor=blue,
    urlcolor=blue
}

\definecolor{dkgreen}{rgb}{0,0.6,0}
\definecolor{gray}{rgb}{0.5,0.5,0.5}
\definecolor{mauve}{rgb}{0.58,0,0.82}

\lstset{frame=tb,
  language=C,
  aboveskip=1mm,
  belowskip=0mm,
  backgroundcolor=\color{black!5},
  showstringspaces=false,
  columns=flexible,
  basicstyle=\footnotesize,
  numbers=none,
  numberstyle=\tiny\color{gray},
  keywordstyle=\color{blue},
  commentstyle=\color{dkgreen},
  stringstyle=\color{mauve},
  breaklines=true,
  breakatwhitespace=true,
  tabsize=3
}

\renewcommand\lstlistingname{Code Block}

\title{%
    McPherson Monochromator Controller Software \\
    User Manual \\
    \small Revision 1}
\author{Mit Bailey}
\date{18 September 2022}

\setcounter{tocdepth}{3}
\setcounter{secnumdepth}{3}

\newcommand*{\fullref}[1]{\hyperref[{#1}]{\ref*{#1} \nameref*{#1}}}

\begin{document}

\maketitle
\newpage

\tableofcontents
\newpage

\section{Getting Started} \label{section:gettingstarted}

\subsubsection{Download}
The latest version of the software will always be available at \url{https://github.com/mitbailey/MMC/releases}. Once on the webpage, extend the \textbf{Assets} drop-down menu and click the \verb|mmc.exe| file to begin downloading.

\subsubsection{Installation}
Currently there is no installation option.

\section{Starting the Program} \label{section:startingtheprogram}

\subsection{Startup} \label{subsection:startup}
To start the program simply double-click the previously downloaded \verb|mmc.exe| file. Please note that on startup the program will attempt to find the configuration file, \verb|config.ini|, in its current directory. If it does not find this file, one will be created with default settings. 


\begin{tcolorbox}
    {\color{red}\textbf{\emph{IMPORTANT!}}} \\
    As of version \href{https://github.com/mitbailey/MMC/releases/tag/v0.3}{v0.3}, the McPherson Monochromator Controller Software (MMCS) requires that all hardware be connected when starting the program. In later implementations, a Device Manager window will allow selection of specific hardware and connection of hardware after MMCS boot-up.
\end{tcolorbox}

Please see \fullref{section:compatiblehardware} for details on compatibility.

\subsection{Debug Startup}
Starting the program with the command line argument \verb|1| will allow the MMCS to run in debug mode. In this mode the program will not attempt to connect to hardware and will simulate all functionality.

\subsection{Startup Behavior}

During initial startup, a loading window will be displayed while the MMCS searches for compatible hardware. Once the MMCS is started and has automatically detected and connected successfully to the hardware, it will begin by homing the device. The program will be mostly uninteractable until this is complete.

If the software is unable to find compatible hardware, the loading window will close and the program will stop running. In future versions a Device Manager window will be displayed allowing the user to select and connect hardware.

\section{Compatible Hardware} \label{section:compatiblehardware}

A list of currently supported hardware can be found in the \emph{Compatible Hardware} section of the README (\url{https://github.com/mitbailey/MMC#readme}).

\section{Graphical User Interface} \label{section:interface}

The Graphical User Interface (GUI) is split into three main areas: the left \emph{Interactable Controls} area, the right \emph{Data Graph} area containing a data display graph, and the bottom \emph{System Status} area. The GUI also has a \emph{Menu Bar} along the top.

\subsection{Interactable Controls Area}

This section is broken up into manual controls, scanning controls, and a data table. 

The manual control section allows the user to enter a desired position in nanometers and then press \verb|Move to Position| to move the monochromator to the desired wavelength. Also present is a \verb|Home| button which manually homes the device.

The scanning control section allows the user to perform a scan with the device. Once \verb|Begin Scan| is pressed, the device will move to 0 nanometers and then to the desired start position. It begins by taking data at the start position, stepping by the distance specified in the \verb|Step| box collecting data at each step, and finishes by taking data at the end position.

Once a scan is completed, it will be displayed in the data table. By default it is also shown in the Data Graph to the right. The user can edit the Name and toggle visibility on the Data Graph by checking the \verb|Plot| checkbox. Selecting a scan and pressing \verb|Save Data| or \verb|Delete Data| will open a save-to-PC prompt or delete the saved data respectively.

The \verb|Machine Configuration| button opens a window allowing the user to manually specify details about the monochromator device they are using.

\subsection{Data Graph Area}

The graph shows all scans by default. Scans can be removed by unchecking the \verb|Plot| checkbox in the Data Table or by pressing the \verb|Clear Plots| button underneath the Data Graph.

\subsection{System Status Area}

This area shows, from left to right, the current monochromator position in nanometers, the current behavior of the system (system status), and the percent complete progress of any current scans.

\subsection{Menu Bar}

The menu bar contains a number of options and helpful links.

\end{document}